% !TEX root = RadCal_User_Guide.tex

\typeout{new file: Structure_code_Chapter.tex}

\chapter{Structure of the Code}
\label{chap:Code}
This chapter presents the structure of the code and the diverse subroutines and functions that comprise it. Originally written in FORTRAN 77, RadCal has been rewritten in Fortran 2008 to benefit from the recent language updates. Furthermore, the code has been divided into modules, subroutine, and functions for ease of code maintenance and understanding.


RadCal source file, \verb=RADCAL.f90=, contains one main module, named \verb=radcal=, and one main program or driver, named \verb=driver=, which uses the \verb=radcal= module. The \verb=radcal= module contains about 40 functions and subroutines that initialize the different species spectral data (mean absorption coefficient $\bar{\kappa}$ and for some species the overlap parameter $\beta$), discretize the spectral range, solve the radiative transfer equation (RTE) for homogeneous and/or non-homogeneous paths, extract the adequate spectral data, interpolate the data when needed, read the input file \verb=RADCAL.in=, deallocate memory prior to exiting RadCal, print some error messages whenever some errors are raised. The \verb=radcal= module is self-consistent. The only dependence is \verb=OpenMP= which is only required when the code is compiled using the makefile \verb-PARALL=1- flag.


The program \verb=driver= contains also one subroutine that creates the TECPLOT file named \verb=<CASEID>.tec=, and plots in ASCII format the wavenumber, the spectral transmissivity, and the incidents spectral intensity. The \verb=driver= program starts by a call to the intrinsic routine \verb=cpu_time= to initialize the system chronometer (it is used to assess the execution time of RadCal) and opens or creates the output file \verb=RADCAL.out=. It then calls the subroutine \verb=read_input= from the \verb=radcal= module to read the input file \verb=RADCAL.in= and populates the needed variables. Then \verb=driver= calls \verb=init_radcal= to initialize the species spectral variables; the scope of these variables is limited to the entire module, they cannot be accessed from the \verb=driver= program as they have the \verb=private= attribute.


Once the data are initialized, the \verb=driver= program calls \verb=sub_radcal= which solves the RTE over the user defined spectral range, calculates and returns the integrated quantities presented in Section~\ref{sec::Output}. The main program then writes in the output file \verb=RADCAL.out= the appropriate data, and calls \verb=tau_print= to print the wavenumber in $\rm cm^{-1}$, the spectral transmissivity in \%, and the incident spectral intensity in $\rm W/m^{2}/str/cm^{-1}$.
Finally the \verb=driver= program calls \verb=rcdealloc= to deallocate the variables allocated in \verb=init_radcal= and calls \verb=cpu_time= to returns the execution time. If the code was compiled using the makefile \verb-DEBUG=1- flag, then RadCal prints the execution time in \textit{ms}, along with the \verb=svn= version of RadCal used, the date at which this version of RadCal has been committed into the \verb=svn= system, and the date at which the code has been compiled. RadCal then stops.

\section{RadCal Module}

\subsection{RadCal Variables}
 \begin{lstlisting}
 character(255), parameter, private :: Radcalid='$Id: RADCAL.f90 51 2014-06-05 18:56:35Z vlecous1 $'
 character(255), parameter, public  :: Radcalrev='$Revision: 51 $'
 character(255), parameter, public  :: Radcaldate='$Date: 2014-06-05 14:56:35 -0400 (Thu, 05 Jun 2014) $'

 character(255), parameter, public :: date_compile = date

 integer, parameter :: fb = selected_real_kind(6)
 integer, parameter :: eb = selected_real_kind(12)

 integer, public :: n_threads

    character(len=32)           :: radcal_id
    character(len=32)           :: id
    character(len=2048)         :: comments
    real(eb), pointer, dimension(:) :: bands
    real(eb), pointer, dimension(:) :: temp_exp

 integer, parameter :: n_species = 16      ! number of radcal species (including soot) plus nitrogen and oxygen
 integer, parameter :: n_species_gas = n_species-1 ! number of radcal gaseous species

 type(elements), dimension(n_species) :: species !

 character(len=255) :: chid  ! radcal case id
 character(len=255) :: title ! radcal case title

 real(eb), allocatable, dimension(:,:) :: gamma, sd15, sd, sd7, sd3

 integer :: n_temp_c3h6,   n_band_c3h6
 integer :: n_temp_c3h8,   n_band_c3h8
 integer :: n_temp_c7h16,  n_band_c7h16
 integer :: n_temp_c7h8,   n_band_c7h8
 integer :: n_temp_ch4,    n_band_ch4
 integer :: n_temp_ch3oh,  n_band_ch3oh
 integer :: n_temp_c5h8o2, n_band_c5h8o2
 integer :: n_temp_c2h6,   n_band_c2h6
 integer :: n_temp_c2h4,   n_band_c2h4

 integer :: i_model_c3h6   ! 1: goody, 2: malkmus, 3: elsasser
 integer :: i_model_c3h8   ! 1: goody, 2: malkmus, 3: elsasser
 integer :: i_model_c7h16  ! 1: goody, 2: malkmus, 3: elsasser
 integer :: i_model_c7h8   ! 1: goody, 2: malkmus, 3: elsasser
 integer :: i_model_ch3oh  ! 1: goody, 2: malkmus, 3: elsasser
 integer :: i_model_c5h8o2 ! 1: goody, 2: malkmus, 3: elsasser
 integer :: i_model_c2h6   ! 1: goody, 2: malkmus, 3: elsasser
 integer :: i_model_c2h4   ! 1: goody, 2: malkmus, 3: elsasser

 integer :: i_model_co2   ! 1: goody, 2: malkmus, 3: elsasser
 integer :: i_model_h2o   ! 1: goody, 2: malkmus, 3: elsasser
 integer :: i_model_co    ! 1: goody, 2: malkmus, 3: elsasser
 integer :: i_model_ch4   ! 1: goody, 2: malkmus, 3: elsasser

 real(eb), allocatable, dimension(:)    :: sd_c3h6_temp
 real(eb), allocatable, dimension(:)    :: sd_c3h8_temp
 real(eb), allocatable, dimension(:)    :: sd_c7h16_temp
 real(eb), allocatable, dimension(:)    :: sd_c7h8_temp
 real(eb), allocatable, dimension(:)    :: sd_ch3oh_temp
 real(eb), allocatable, dimension(:)    :: sd_c5h8o2_temp
 real(eb), allocatable, dimension(:)    :: sd_c2h6_temp
 real(eb), allocatable, dimension(:)    :: sd_ch4_temp
 real(eb), allocatable, dimension(:)    :: sd_c2h4_temp

 real(eb), allocatable, dimension(:)    :: be_c3h6
 real(eb), allocatable, dimension(:)    :: be_c3h8
 real(eb), allocatable, dimension(:)    :: be_c7h16
 real(eb), allocatable, dimension(:)    :: be_c7h8
 real(eb), allocatable, dimension(:)    :: be_ch3oh
 real(eb), allocatable, dimension(:)    :: be_c5h8o2
 real(eb), allocatable, dimension(:)    :: be_c2h6
 real(eb), allocatable, dimension(:)    :: be_c2h4

 real(eb), allocatable, target, dimension(:,:)  :: om_bnd_ch4
 real(eb), allocatable, target, dimension(:,:)  :: om_bnd_c3h6
 real(eb), allocatable, target, dimension(:,:)  :: om_bnd_c3h8
 real(eb), allocatable, target, dimension(:,:)  :: om_bnd_c7h16
 real(eb), allocatable, target, dimension(:,:)  :: om_bnd_c7h8
 real(eb), allocatable, target, dimension(:,:)  :: om_bnd_ch3oh
 real(eb), allocatable, target, dimension(:,:)  :: om_bnd_c5h8o2
 real(eb), allocatable, target, dimension(:,:)  :: om_bnd_c2h6
 real(eb), allocatable, target, dimension(:,:)  :: om_bnd_c2h4

 real(eb), allocatable, target, dimension(:,:)  :: sd1_ch4
 real(eb), allocatable, target, dimension(:,:)  :: sd2_ch4

 real(eb), allocatable, target, dimension(:,:)  :: sd1_c3h6
 real(eb), allocatable, target, dimension(:,:)  :: sd2_c3h6
 real(eb), allocatable, target, dimension(:,:)  :: sd3_c3h6

 real(eb), allocatable, target, dimension(:,:)  :: sd1_c3h8
 real(eb), allocatable, target, dimension(:,:)  :: sd2_c3h8

 real(eb), allocatable, target, dimension(:,:)  :: sd1_c7h16
 real(eb), allocatable, target, dimension(:,:)  :: sd2_c7h16

 real(eb), allocatable, target, dimension(:,:)  :: sd1_c7h8
 real(eb), allocatable, target, dimension(:,:)  :: sd2_c7h8
 real(eb), allocatable, target, dimension(:,:)  :: sd3_c7h8
 real(eb), allocatable, target, dimension(:,:)  :: sd4_c7h8
 real(eb), allocatable, target, dimension(:,:)  :: sd5_c7h8

 real(eb), allocatable, target, dimension(:,:)  :: sd1_ch3oh
 real(eb), allocatable, target, dimension(:,:)  :: sd2_ch3oh
 real(eb), allocatable, target, dimension(:,:)  :: sd3_ch3oh
 real(eb), allocatable, target, dimension(:,:)  :: sd4_ch3oh

 real(eb), allocatable, target, dimension(:,:)  :: sd1_c5h8o2
 real(eb), allocatable, target, dimension(:,:)  :: sd2_c5h8o2
 real(eb), allocatable, target, dimension(:,:)  :: sd3_c5h8o2
 real(eb), allocatable, target, dimension(:,:)  :: sd4_c5h8o2
 real(eb), allocatable, target, dimension(:,:)  :: sd5_c5h8o2
 real(eb), allocatable, target, dimension(:,:)  :: sd6_c5h8o2

 real(eb), allocatable, target, dimension(:,:)  :: sd1_c2h6
 real(eb), allocatable, target, dimension(:,:)  :: sd2_c2h6
 real(eb), allocatable, target, dimension(:,:)  :: sd3_c2h6

 real(eb), allocatable, target, dimension(:,:)  :: sd1_c2h4
 real(eb), allocatable, target, dimension(:,:)  :: sd2_c2h4
 real(eb), allocatable, target, dimension(:,:)  :: sd3_c2h4
 real(eb), allocatable, target, dimension(:,:)  :: sd4_c2h4

 real(eb), allocatable, target, dimension(:,:)  :: gammad1_c3h6
 real(eb), allocatable, target, dimension(:,:)  :: gammad2_c3h6
 real(eb), allocatable, target, dimension(:,:)  :: gammad3_c3h6

 real(eb), allocatable, target, dimension(:,:)  :: gammad1_c3h8
 real(eb), allocatable, target, dimension(:,:)  :: gammad2_c3h8

 real(eb), allocatable, target, dimension(:,:)  :: gammad1_c7h16
 real(eb), allocatable, target, dimension(:,:)  :: gammad2_c7h16

 real(eb), allocatable, target, dimension(:,:)  :: gammad1_c7h8
 real(eb), allocatable, target, dimension(:,:)  :: gammad2_c7h8
 real(eb), allocatable, target, dimension(:,:)  :: gammad3_c7h8
 real(eb), allocatable, target, dimension(:,:)  :: gammad4_c7h8
 real(eb), allocatable, target, dimension(:,:)  :: gammad5_c7h8

 real(eb), allocatable, target, dimension(:,:)  :: gammad1_ch3oh
 real(eb), allocatable, target, dimension(:,:)  :: gammad2_ch3oh
 real(eb), allocatable, target, dimension(:,:)  :: gammad3_ch3oh
 real(eb), allocatable, target, dimension(:,:)  :: gammad4_ch3oh

 real(eb), allocatable, target, dimension(:,:)  :: gammad1_c5h8o2
 real(eb), allocatable, target, dimension(:,:)  :: gammad2_c5h8o2
 real(eb), allocatable, target, dimension(:,:)  :: gammad3_c5h8o2
 real(eb), allocatable, target, dimension(:,:)  :: gammad4_c5h8o2
 real(eb), allocatable, target, dimension(:,:)  :: gammad5_c5h8o2
 real(eb), allocatable, target, dimension(:,:)  :: gammad6_c5h8o2

 real(eb), allocatable, target, dimension(:,:)  :: gammad1_c2h6
 real(eb), allocatable, target, dimension(:,:)  :: gammad2_c2h6
 real(eb), allocatable, target, dimension(:,:)  :: gammad3_c2h6

 real(eb), allocatable, target, dimension(:,:)  :: gammad1_c2h4
 real(eb), allocatable, target, dimension(:,:)  :: gammad2_c2h4
 real(eb), allocatable, target, dimension(:,:)  :: gammad3_c2h4
 real(eb), allocatable, target, dimension(:,:)  :: gammad4_c2h4

 real(eb), dimension(:,:), pointer, private :: kappa_d, gamma_d
 real(eb), dimension(:),   pointer, private :: band_omega


 real(eb), allocatable, dimension(:)  :: ab, wave_number, lambda
 real(eb), allocatable, dimension(:)  :: incident_radiance

 real(eb), allocatable, dimension(:,:) :: ttau

 real(eb), allocatable, dimension(:,:) :: partial_pressures_atm
 real(eb), allocatable, dimension(:)   :: temp_gas
 real(eb), allocatable, dimension(:)   :: segment_length_m
 real(eb), allocatable, dimension(:)   :: total_pressure_atm

 real(eb) :: ommin, ommax, lambdamin, lambdamax
 real(eb) :: twall, tau, xtot, fv

 integer :: nom, npt


 real(eb), parameter :: sigma   = 5.670400e-8_eb
 real(eb), parameter :: m_to_cm = 100.0_eb  ! convertion factor meters -> centimeters
 real(eb), parameter :: cm_to_m = 0.01_eb   ! convertion factor centimeters -> meters

 character(30) :: radcal_id
 real(eb) :: pfuel

 real(eb), parameter :: zero_p    = tiny(sigma)
 real(eb), parameter :: pi        = acos(-1.0)
 real(eb), parameter :: sqrtpi    = sqrt(pi)
 real(eb), parameter :: rpi_sigma = sigma/pi

 integer :: i_co, i_co2, i_h2o, i_n2, i_o2, i_fv, i_c2h4, i_c2h6, i_c3h6, i_c3h8
 integer :: i_c7h8, i_c7h16, i_ch3oh, i_ch4, i_ch4_old, i_mma


\end{lstlisting}

Note: variables \verb=sd_<species>_temp= contain the temperature corresponding to the experimental conditions at which the mean absorption coefficient \verb=sd<band index>_<species>= (example: \verb=sd1_ch4=) and the band overlap parameter \verb=gammad<band index>_<species>= (example: \verb=gammad1_c7h8=) where obtained from. The variables \verb=om_bnd_<species>= contain the lower, the upper bounds, and the narrow band size (in wavenumber) of a species main band.

\subsection{Subroutine init\_radcal}
\label{sub:init_radcal}
\begin{lstlisting}
subroutine init_radcal(io)
\end{lstlisting}

Initializes variables \verb=nom=, vectors \verb=wave_number= and \verb=lambda=, and calls \verb=rcalloc= based on parameters present in input file \verb=radcal.in=.

\begin{lstlisting}
 integer, intent(in) :: io
 real(eb) :: dom, omega
 integer :: i_wavenumb
\end{lstlisting}


\subsection{Subroutine sub\_radcal}
\label{sub:sub_radcal}
\begin{lstlisting}
subroutine sub_radcal(effective_absorption,planck_mean_absorption,radiance,
total_transmissivity,io)
\end{lstlisting}
input variables:
\begin{itemize}
 \item io: (integer) output file unit
\end{itemize}
output variables:
\begin{itemize}
 \item \verb=effective_absorption=   : (real) effective absorption coefficient, in $\rm cm^{-1}$, as defined by Eq.~\ref{eq:effective_epsilon}
 \item \verb=planck_mean_absorption= : (real) planck mean absorption coefficient, in $\rm cm^{-1}$, as defined by Eq.~\ref{eq::planck_mean}
 \item \verb=radiance=               : (real) incident power per unit of area per                                  unit of solid angle, units: $\rm W/m^{2}/str/cm^{-1}$, as defined by Eq.~\ref{eq::received_flux}
 \item \verb=total_transmissivity=   : (real) total transmissivity, as defined by Eq.~\ref{eq::total_transmissivity}, computed only when \verb=TWALL= (blackbody wall temperature $T_{w}$) is strictly greater than 0. Dimensionless number
\end{itemize}
Local variables:
\begin{itemize}
 \item \verb=a_collision= : (real) collision broadened fine structure parameter, as defined by Eq.~\ref{eq::beta}
 \item \verb=a_doppler=   : (real) doppler broadened fine structure parameter, as defined by Eq.~\ref{eq::beta}
 \item \verb=azotemp=     : (real) ratio reference temperature (273~K) over local temperature, $= 273/T$
 \item \verb=omega=       : (real) wavenumber
 \item \verb=optical_thickness=: (real) optical thickness of the ith species,  units: cmstp
 \item \verb=ptot=        : (real) total pressure in atm
 \item \verb=temp4=       : (real) local temperature raised to the power 4
 \item \verb=x_collision= : (real) optical depth for a pure collision curve of growth, as defined by Eq.~\ref{eq::Collision_optical_depth}
 \item \verb=x_doppler=   : (real) optical depth for a pure doppler curve of growth, as defined by Eqs.~\ref{eq::Doppler_optical_path_Goody} and \ref{eq::Doppler_optical_path_Malkmus}
 \item \verb=x_particle=  : (real) optical depth for particle, as defined by Eq.~\ref{eq::soot_optical_depth}
\end{itemize}
This is the most important subroutine of the RadCal module. This function solves the RTE as defined by Eqs.~\ref{eq::RTE_wavenumber_H} and \ref{eq::RTE_wavenumber_NH} and returns the effective absorption coefficient, the Planck mean absorption coefficient, the total transmissivity, and populates the global variables \verb=incident_radiance= (spectral incident intensity as defined by Eq.~\ref{eq::received_flux}) and \verb=ttau= (spectral transmissivities used in Eq.~\ref{eq::RTE_wavenumber_NH}).

\begin{lstlisting}
 integer, intent(in) :: io
 real(eb), intent(out) :: effective_absorption, planck_mean_absorption,radiance, total_transmissivity

 real(eb), allocatable, dimension(:)   :: path_length_cm, azotemp
 real(eb), allocatable, dimension(:)   :: taus, taul, x_particle, ab_planck,ab_tau, bb_spect, bb_gas, bb_wall
 real(eb), allocatable, dimension(:,:) :: curtis_xstar, curtis_acollision,curtis_adoppler, gc,optical_thickness, optical_depth
 real(eb) :: rsl, rss, omega, wave_length,  dambda, lterm, temp4,avg_temp, total_length_cm, int_bb_gas, int_bb_wall
 integer :: i_wavenumb, kmax, kmin, i_path, i_species
\end{lstlisting}

\subsection{Function collision\_broadening}
\label{fun:collision_broadening}
\begin{lstlisting}
function collision_broadening(species_pressure_atm,ptot,azotemp) result(gc)
\end{lstlisting}
 This function computes the collision broadening half-width at half-maximum (HWHM). Calculation proceeded in accordance with the SLG model, Table 5-18, in nasa sp-3080, based on eq 5-34 from Ref.~\cite{Ludwig1973}, also presented in Eq.~\ref{eq:gamma_L}. It returns vector \verb=gc= that contains the collision broadening HWHM including foreign and self-broadening collision coefficients. \verb=gc= units in $\rm cm^{-1}$.

\begin{lstlisting}
 real(eb), dimension(:), intent(in) :: species_pressure_atm ! species partial pressure, units: atm
 real(eb), intent(in) :: ptot         ! total pressure, units: atm
 real(eb), intent(in) :: azotemp      ! recall azotemp = 273/T, units: none

 real(eb), allocatable, dimension(:) :: gc ! units in cm-1
 real(eb) :: p_fuel
 integer, dimension(6) :: i_indices
 integer :: i, ii, i_fuel
\end{lstlisting}


\subsection{Subroutine species\_optical\_depth}
\label{sub:species_optical_depth}

\begin{lstlisting}
subroutine species_optical_depth(path_length_cm,p_atm,ptot,gc,temp,omega,optical_thickness,
curtis_xstar,curtis_acollision, curtis_adoppler,optical_depth)
\end{lstlisting}
This subroutine computes the species optical depth for all the species given a wavenumber \verb=omega= and species optical thickness.
It returns the optical depth for each species using the Curtis Godson approximation. Follows method presented in Sections~\ref{sec::homogeneous_path} and \ref{sec::Curtis_Godson}.

Arguments in:
\begin{itemize}
 \item \verb=path_length_cm= : scalar, physical length, in cm
 \item \verb=p_atm=           : vector, (dimension \verb=nspecies=) species partial pressure, in atm
 \item \verb=ptot=             : scalar, total pressure, in atm
 \item \verb=gc=               : vector, collision-broadened half-width values, in $\rm cm^{-1}$
 \item \verb=temp=             : scalar, local temperature, in K
 \item \verb=omega=            : scalar, wavenumber, in $\rm cm^{-1}$
 \item \verb=optical_thickness=: vector (dimension \verb=nspecies=) elements, units in stp.cm.atm
\end{itemize}

Arguments inout:
 variables used to compute the Curtis-Godson approximation parameters. The first value (\textit{i.e.}) \verb=(nspecies,1)= was either computed previously or is equal to 0 (for the first point).
\begin{itemize}
 \item \verb=curtis_xstar=      : array of values of $\bar{\kappa}U$. Dimension: \verb=(nspecies,2)=
 \item \verb=curtis_acollision= : array of values of $\beta$. Dimension: \verb=(nspecies,2)=
 \item \verb=curtis_adoppler=   : array of values of $\gamma_D/d$. Dimension: \verb=(nspecies,2)=
 \item \verb=optical_depth=     : array of values of elements optical depth, dimensionless, dimension: \verb=(nspecies,2)=.
\end{itemize}

\begin{lstlisting}
 real(eb), dimension(:), intent(in) :: p_atm, optical_thickness, gc
 real(eb), intent(in) :: ptot, temp, omega, path_length_cm

 real(eb), dimension(:,:), intent(inout) :: curtis_xstar, curtis_acollision,curtis_adoppler, optical_depth
 real(eb) :: x_doppler, x_collision, xstar
 real(eb) :: sdweak,    a_collision, a_doppler
 integer  :: i_model ! model used for snb fitting: 1: goody, 2 malkmus, 3 elsasser
 integer  :: i
\end{lstlisting}

\subsection{Function growth\_doppler}

\begin{lstlisting}
elemental function growth_doppler(xstar,a_doppler,i_model) result(x_doppler)
\end{lstlisting}

This function computes the Doppler curve of growth. Solves either Eq.~\ref{eq::Doppler_optical_path_Goody} (if i\_model=1 or 3) or Eq.~\ref{eq::Doppler_optical_path_Malkmus} (for i\_model=2).

Variables in:
\begin{itemize}
 \item \verb=xstar=: optical pathlength * mean absorption coefficient, $\bar{\kappa}U$
 \item \verb=a_doppler=: Doppler fine structure parameter, $\gamma_D/d$
 \item \verb=i_model=: model used for SNB fitting: 1: Goody, 2 Malkmus, 3 Elsasser
\end{itemize}

Variable out:
\begin{itemize}
 \item \verb=x_doppler=.
\end{itemize}

\begin{lstlisting}
real(eb), intent(in) :: xstar
real(eb), intent(in) :: a_doppler
integer,  intent(in) :: i_model

real(eb) :: x_doppler
\end{lstlisting}

\subsection{Function combined\_lines}

\begin{lstlisting}
elemental function combined_lines(x_collision, x_doppler, xstar) result(optical_depth)
\end{lstlisting}

Computes the combined collision (Lorentz) and Doppler optical depths, and the optical depth based on Eqs.~\ref{eq::combined_optical_path} and \ref{eq::trans_HPP_final}. See Section~\ref{sec::homogeneous_path}.

Variables in:
\begin{itemize}
 \item \verb=x_collision=: Lorentz curve of growth
 \item \verb=x_doppler=: Doppler curve of growth
 \item \verb=xstar=: optical pathlength * mean absorption coefficient
\end{itemize}

Variable out:
\begin{itemize}
 \item \verb=optical_depth=.
\end{itemize}

\begin{lstlisting}
real(eb), intent(in) :: x_collision
real(eb), intent(in) :: x_doppler
real(eb), intent(in) :: xstar

real(eb) :: optical_depth
real(eb) :: y_doppler, y_collision, y_combined
\end{lstlisting}

\subsection{Function Goody}

\begin{lstlisting}
elemental function goody(xstar,a_collision) result(x_goody)
\end{lstlisting}
 Function computes the equivalent line width over the average line spacing using the goody statistical model, as defined by Eq.~\ref{eq::Goody}.

 Variables in:
 \begin{itemize}
  \item \verb=xstar=: optical pathlength * mean absorption coefficient
  \item \verb=a_collision= : fine structure parameter or overlap parameter $\beta$, see Eq.~\ref{eq::beta}.
  \end{itemize}

Returns \verb=x_goody=.

\begin{lstlisting}
real(eb), intent(in) :: xstar
real(eb), intent(in) :: a_collision

real(eb)  :: x_goody
\end{lstlisting}


\subsection{Function Malkmus}

\begin{lstlisting}
elemental function malkmus(xstar,a_collision) result(x_malkmus)
\end{lstlisting}
Function computes the equivalent line width over the average line spacing, using the Malkmus statistical model. See Eq.~\ref{eq::Malkmus}.

Variables in:
\begin{itemize}
 \item \verb=xstar=: optical pathlength * mean absorption coefficient
  \item \verb=a_collision= : fine structure parameter or overlap parameter $\beta$, see Eq.~\ref{eq::beta}
\end{itemize}

Returns \verb=x_malkmus= (optical depth for a pure collision curve).

\begin{lstlisting}
real(eb), intent(in) :: xstar
real(eb), intent(in) :: a_collision

real(eb)  :: x_malkmus
\end{lstlisting}

\subsection{Function Elsasser}

\begin{lstlisting}
elemental function elsasser(xstar,a_collision) result(x_elsasser)
\end{lstlisting}

This function computes the equivalent line width over the average line spacing using the Elsasser statistical model, see Eq.~\ref{eq::Elsasser}.

Variables in:
\begin{itemize}
 \item \verb=xstar=: optical pathlength * mean absorption coefficient
 \item \verb=a_collision= : fine structure parameter or overlap parameter $\beta$, see Eq.~\ref{eq::beta}
\end{itemize}

Returns \verb=x_elsasser=.

\begin{lstlisting}
real(eb), intent(in) :: xstar
real(eb), intent(in) :: a_collision

real(eb)  :: x_elsasser
real(eb) :: x_collision
\end{lstlisting}

\subsection{Function partition\_function}

\begin{lstlisting}
pure function partition_function(transition_wn,degeneracy,temp)
\end{lstlisting}

This function calculates the partition function of a harmonic quantum oscillator knowing the different frequencies and degeneracies.

\begin{lstlisting}
real(eb), dimension(:), intent(in) :: transition_wn ! Wavenumber associated with

integer, dimension(:), intent(in) :: degeneracy  ! Denegeracy of energy level

real(eb), intent(in) :: temp                     ! Temperature in Kelvin

real(eb) :: partition_function

real(eb), parameter :: q2 = 1.4388_eb   ! speed of light*planck cns/boltzmann
real(eb) :: q2_over_T
integer  :: n_mode
integer  :: i_mode
\end{lstlisting}


\subsection{Function approx\_vib\_rot}

\begin{lstlisting}
pure subroutine approx_vib_rot(transition_matrix,line_strength,om,temp,omega,  &
                               bcnt,be,gc1, gd, gdinv,gddinv,sdweak,line_spacing)
\end{lstlisting}

\begin{lstlisting}
integer, dimension(:,:), intent(in) :: transition_matrix ! Square matrix
real(eb), dimension(:),   intent(in) :: bcnt ! Band center in wavenumbers (cm-1)
real(eb), dimension(:),   intent(in) :: om
real(eb), dimension(:),   intent(in) :: line_strength

real(eb), intent(in) :: omega ! Wavenumber of sought properties
real(eb), intent(in) :: be    ! Rotational constant
real(eb), intent(in) :: temp  ! Temperature in Kelvin
real(eb), intent(in) :: gc1
real(eb), intent(in) :: gd
real(eb), intent(in), optional :: line_spacing ! Optional

real(eb), intent(out) :: sdweak
real(eb), intent(out) :: gdinv
real(eb), intent(out) :: gddinv

real(eb), dimension(:), allocatable :: transition_wn, atot
real(eb), parameter :: q2 = 1.4388_eb   ! speed of light*planck cns/boltzmann
real(eb), parameter :: t0 = 300._eb     ! reference temperature, in kelvin

real(eb) :: t0ot
real(eb) :: q2ot
real(eb) :: dinv

integer  :: n_transitions
integer  :: n_vibrations
integer :: i_transition
\end{lstlisting}

\subsection{Subroutine co2}

\begin{lstlisting}
elemental subroutine co2(omega,temp,gc1,sdweak,gdinv,gddinv,i_model)
\end{lstlisting}
Given \verb=omega= (wavenumber in $\rm cm^{-1}$), \verb=temp= (temperature in K), and \verb=gc1= (half width at half maximum for Lorentz line, in $\rm cm^{-1}$), this subroutine returns the spectral properties of $\rm CO_2$: spectral mean absorption coefficient (\verb=sdweak=), the collision broadening fine structure parameter (\verb=gdinv=), the Doppler broadening fine structure parameter (\verb=gddinv=), and the narrow band model used (\verb=i_model=).

\begin{lstlisting}
real(eb), intent(in) :: omega ! wavenumber, units in cm-1
real(eb), intent(in) :: temp  ! temperature, units in kelvin
real(eb), intent(in) :: gc1   ! half width at half maximum for lorentz line, units in cm-1

real(eb), intent(out):: sdweak  ! narrow band mean absortion coefficient, units in cm-1
real(eb), intent(out):: gdinv   ! line width to line spacing ration for lorentz lines
real(eb), intent(out):: gddinv  ! line width to line spacing ration for doppler
integer,  intent(out):: i_model ! model used for snb fitting: 1: goody, 2 malkmus, 3 elsasser

integer  :: i,j,k,l

real(eb), parameter :: wm_co2 = 44._eb      ! molecular weight (g/mol)
real(eb), parameter :: be     = 0.391635_eb ! rotational constant co2
real(eb), parameter :: q2     = 1.4388_eb   ! speed of light*planck cns/boltzmann
real(eb), parameter :: t0     = 300._eb     ! reference temperature, in kelvin

real(eb), parameter :: om1 = 1354.91_eb  ! fundamental frequency for transition (000)->(100), units cm-1
real(eb), parameter :: om2 =  673.0_eb   ! fundamental frequency for transition (000)->(010), units cm-1
real(eb), parameter :: om3 = 2396.49_eb  ! fundamental frequency for transition (000)->(001), units cm-1

real(eb), dimension(3) :: atot, bcnt
real(eb), dimension(:), allocatable  :: om, line_strength
integer, dimension(:,:), allocatable :: transition_matrix

real(eb) :: aa,bb,cc,qq,ee,ff,gg,sminus,splus,sdstrg,gd,com1,com2,
com3,x13,x23,x33,xbar,om12,alpha,omprim,v3,gam,omvv3,delta,v,omvbar,
f1,f2,unflo1,unflo2,unflo3,test,vbar1,oma,omb,ttemp,tt,
t1,tw,ww,temp1,temp2,temp3,dinv,a,b,d,g,w1,dinv1,dinv2,dinv3,q2ot,t0ot,q2ot0
\end{lstlisting}

\subsection{Subroutine h2o}

\begin{lstlisting}
elemental subroutine h2o(omega,temp,gc2,sdweak,gdinv,gddinv,i_model)
\end{lstlisting}
Given \verb=omega= (wavenumber in $\rm cm^{-1}$), \verb=temp= (temperature in K), and \verb=gc2= (half width at half maximum for Lorentz line, in $\rm cm^{-1}$), this subroutine returns the spectral properties of $\rm H_2O$: spectral mean absorption coefficient (\verb=sdweak=), the collision broadening fine structure parameter (\verb=gdinv=), the Doppler broadening fine structure parameter (\verb=gddinv=), and the narrow band model used (\verb=i_model=).

\begin{lstlisting}
real(eb), intent(in) :: omega ! wavenumber,  in cm-1
real(eb), intent(in) :: temp  ! temperature, in kelvin
real(eb), intent(in) :: gc2   ! half width at half maximum, in cm-1

real(eb), intent(out) :: sdweak  ! spectral mean absorption coefficient, in cm-1
real(eb), intent(out) :: gddinv  ! line width to line spacing ration for doppler
real(eb), intent(out) :: gdinv   ! line width to line spacing ratio for lorentz boradening
integer,  intent(out) :: i_model ! model used for snb fitting: 1: goody, 2 malkmus, 3 elsasser

integer  :: i,j
real(eb) :: w1, ww, t1, tt, tw, d, b, dinv, ttemp, gd
real(eb), parameter :: wm_h2o = 18._eb ! molecular weight (g/mol)
\end{lstlisting}

\subsection{Subroutine co}
\label{sub:co}

\begin{lstlisting}
subroutine co(omega,temp,gc4,sdweak,gdinv,gddinv,i_model)
\end{lstlisting}
 Given \verb=omega= (wavenumber in $\rm cm^{-1}$), \verb=temp= (temperature in K), and \verb=gc4= (half width at half maximum for Lorentz line, in $\rm cm^{-1}$), this subroutine returns the spectral properties of CO: spectral mean absorption coefficient (\verb=sdweak=), the collision broadening fine structure parameter (\verb=gdinv=), the Doppler broadening fine structure parameter (\verb=gddinv=), and the narrow band model used (\verb=i_model=).

\begin{lstlisting}
integer j
real(eb) omega,temp,gc4,sdweak,gdinv,gddinv,aa,bb,cc,qq,
ee,ff,gg,sminus,splus,sdstrg,b,alpha,a,ome,wx,wy,omprim,
t0,q2,wm,gd,v,gam,omv,delta,d,omvbar,f1,f2,test,
dinv,q2ot,toaz

integer, intent(out) :: i_model
\end{lstlisting}

\subsection{Subroutine pod}
\label{sub:pod}

\begin{lstlisting}
elemental subroutine pod(omega,soot_volume_frac,path_length_cm,temp,x_particle)
\end{lstlisting}
This subroutine calculates the particle optical depth, \verb=x_particle=, of the soot particles present. Eq.~\ref{eq::soot_optical_depth} is used.

Argument in:
\begin{itemize}
 \item \verb=omega=: wavenumber, units: $\rm cm^{-1}$
 \item \verb=soot_volume_frac=: soot volume fraction. no units
 \item \verb=path_length_cm=: physical path length. units in $\rm cm$
 \item \verb=temp=: temperature in Kelvin
\end{itemize}

Argument out:
\begin{itemize}
\item \verb=x_particle=: optical depth due to the presence of soot particle. No units.
\end{itemize}

\begin{lstlisting}
real(eb), intent(in) :: omega, path_length_cm, temp, soot_volume_frac

real(eb), intent(out) :: x_particle

real(eb) :: abco,ff,lambda,rin,rik
\end{lstlisting}


\subsection{Subroutine ch4\_old}
\label{sub:ch4_old}

\begin{lstlisting}
subroutine ch4_old(omega,temp,pch4,ptot,gc3,sdweak,gdinv,gddinv,i_model)
\end{lstlisting}
Given \verb=omega= (wavenumber in $\rm cm^{-1}$), \verb=temp= (temperature in K), and \verb=gc3= (half width at half maximum for Lorentz line, in $\rm cm^{-1}$), this subroutine returns the spectral properties of $\rm CH_4$ using original RadCal data: spectral mean absorption coefficient (\verb=sdweak=), the collision broadening fine structure parameter (\verb=gdinv=), the Doppler broadening fine structure parameter (\verb=gddinv=), and the narrow band model used (\verb=i_model=).

\begin{lstlisting}
integer i,j
real(eb) omega,temp,pch4,ptot,gc3,sdweak,gdinv,gddinv,be,q2,wm,
gd,om1,om2,om3,om4,com1,com2,com3,com4,dinv,pe,w1,sdb,sda,sdc,q2ot,azot,toaz

integer, intent(out) :: i_model

real(eb), dimension(4) :: atot, bcnt
\end{lstlisting}


\subsection{Subroutine ch4}
\label{sub:ch4}

\begin{lstlisting}
subroutine ch4(omega,temp,pch4,ptot,gc3,sdweak,gdinv,gddinv,i_model)
\end{lstlisting}
Given:
\begin{itemize}
 \item \verb=omega=: wavenumber in $\rm cm^{-1}$,
 \item \verb=temp=: temperature in K,
 \item \verb=pch4=: methane partial pressure in atm,
 \item \verb=ptot=: total pressure in atm,
 \item \verb=gc3=: collisional broadening half-width at half-maximum, units in $\rm cm^{-1}$,
\end{itemize}
this subroutine returns the spectral properties of $\rm CH_4$ using new methane data:
\begin{itemize}
 \item \verb=sdweak=: spectral mean absorption coefficient, units in $\rm atm^{-1}.cm^{-1}$,
 \item \verb=gdinv=: the collision broadening fine structure parameter, dimensionless,
 \item \verb=gddinv=: the Doppler broadening fine structure parameter, dimensionless,
 \item \verb=i_model=: the narrow band model used.
\end{itemize}

\begin{lstlisting}
real(eb), intent(in)  :: omega, temp, pch4, ptot, gc3
real(eb), intent(out) :: sdweak, gdinv, gddinv

integer, intent(out)  :: i_model

real(eb) :: gd, pressure_effective, q2ot, azot, toaz, fact1
real(eb) :: dinv_ch4

real(eb), parameter :: q2     = 1.4388_eb ! q2 = speed of light*planck cns/boltzmann
real(eb), parameter :: wm_ch4 = 16.0425_eb ! molecular weight (g/mol)

real(eb), parameter :: be_ch4 = 5.248_eb   ! rotational constants [cm$^{-1}$]

real(eb), dimension(4), parameter ::om_ch4  = (/2914.2_eb, 1526.0_eb, 3020.3_eb, 1306.2_eb/),
com_ch4 = (/1526.0_eb+2._eb*1306.2_eb, 2914.2_eb+1306.2_eb, 3020.3_eb+1306.2_eb, 1526.0_eb+3020.3_eb/),
s2_ch4  = (/0.64_eb,17.6_eb,14.8_eb,5.04_eb/)

real(eb), dimension(4) :: atot

integer i

\end{lstlisting}


\subsection{Subroutine c3h6}
\label{sub:c3h6}

\begin{lstlisting}
subroutine c3h6(omega,temp,pc3h6,ptot,sdweak,gdinv,gddinv,i_model)
\end{lstlisting}
Given:
\begin{itemize}
 \item \verb=omega=: wavenumber in $\rm cm^{-1}$,
 \item \verb=temp=: temperature in K,
 \item \verb=pc3h6=: propylene partial pressure in atm,
 \item \verb=ptot=: total pressure in atm,
 \item \verb=gc3=: collisional broadening half-width at half-maximum, units in $\rm cm^{-1}$,
\end{itemize}
this subroutine returns the spectral properties of propylene using new data:
\begin{itemize}
 \item \verb=sdweak=: spectral mean absorption coefficient, units in $\rm atm^{-1}.cm^{-1}$,
 \item \verb=gdinv=: the collision broadening fine structure parameter, dimensionless,
 \item \verb=gddinv=: the Doppler broadening fine structure parameter, dimensionless,
 \item \verb=i_model=: the narrow band model used.
\end{itemize}

\begin{lstlisting}
real(eb), intent(in)  :: omega, temp, pc3h6, ptot
real(eb), intent(out) :: sdweak, gdinv, gddinv

integer, intent(out) :: i_model

real(eb) :: q2ot, azot, toaz
real(eb) :: gd, pressure_effective
real(eb) :: dinv_c3h6

real(eb), parameter :: q2       = 1.4388_eb  ! q2 = speed of light*planck cns/boltzmann
real(eb), parameter :: wm_c3h6  = 42.0797_eb ! molecular weight (g/mol)

integer :: i_band, i

logical :: in_band ! true if omega is within some tabulated band, false otherwise
\end{lstlisting}


\subsection{Subroutine c3h8}
\label{sub:c3h8}
\begin{lstlisting}
subroutine c3h8(omega,temp,pc3h8,ptot,sdweak,gdinv,gddinv,i_model)
\end{lstlisting}

Given:
\begin{itemize}
 \item \verb=omega=: wavenumber in $\rm cm^{-1}$,
 \item \verb=temp=: temperature in K,
 \item \verb=pc3h8=: propane partial pressure in atm,
 \item \verb=ptot=: total pressure in atm,
 \item \verb=gc3=: collisional broadening half-width at half-maximum, units in $\rm cm^{-1}$,
\end{itemize}
this subroutine returns the spectral properties of propane using new data:
\begin{itemize}
 \item \verb=sdweak=: spectral mean absorption coefficient, units in $\rm atm^{-1}.cm^{-1}$,
 \item \verb=gdinv=: the collision broadening fine structure parameter, dimensionless,
 \item \verb=gddinv=: the Doppler broadening fine structure parameter, dimensionless,
 \item \verb=i_model=: the narrow band model used.
\end{itemize}

\begin{lstlisting}
real(eb), intent(in)  :: omega, temp, pc3h8, ptot
real(eb), intent(out) :: sdweak, gdinv, gddinv

integer, intent(out) :: i_model

real(eb) :: q2ot, azot, toaz
real(eb) :: gd, pressure_effective
real(eb) :: dinv_c3h8

real(eb), parameter :: q2       = 1.4388_eb  ! q2 = speed of light*planck cns/boltzmann
real(eb), parameter :: wm_c3h8  = 44.0956_eb ! molecular weight (g/mol)

integer :: i_band, i

logical :: in_band ! true if omega is within some tabulated band, false otherwise
\end{lstlisting}


\subsection{Subroutine c7h16}
\label{sub:c7h16}

\begin{lstlisting}
subroutine c7h16(omega,temp,pc7h16,ptot,sdweak,gdinv,gddinv,i_model)
\end{lstlisting}

Given:
\begin{itemize}
 \item \verb=omega=: wavenumber in $\rm cm^{-1}$,
 \item \verb=temp=: temperature in K,
 \item \verb=pc7h16=: heptane partial pressure in atm,
 \item \verb=ptot=: total pressure in atm,
 \item \verb=gc3=: collisional broadening half-width at half-maximum, units in $\rm cm^{-1}$,
\end{itemize}
this subroutine returns the spectral properties of heptane using new data:
\begin{itemize}
 \item \verb=sdweak=: spectral mean absorption coefficient, units in $\rm atm^{-1}.cm^{-1}$,
 \item \verb=gdinv=: the collision broadening fine structure parameter, dimensionless,
 \item \verb=gddinv=: the Doppler broadening fine structure parameter, dimensionless,
 \item \verb=i_model=: the narrow band model used.
\end{itemize}

\begin{lstlisting}
real(eb), intent(in)  :: omega, temp, pc7h16, ptot
real(eb), intent(out) :: sdweak, gdinv, gddinv

integer, intent(out) :: i_model

real(eb) :: q2ot, azot, toaz
real(eb) :: gd, pressure_effective
real(eb) :: dinv_c7h16

real(eb), parameter :: q2       = 1.4388_eb  ! q2 = speed of light*planck cns/boltzmann
real(eb), parameter :: wm_c7h16 = 100.2019_eb ! molecular weight (g/mol)

integer :: i_band, i

logical :: in_band ! true if omega is within some tabulated band, false otherwise

\end{lstlisting}


\subsection{Subroutine c7h8}
\label{sub:c7h8}
\begin{lstlisting}
subroutine c7h8(omega,temp,pc7h8,ptot,sdweak,gdinv,gddinv,i_model)
\end{lstlisting}

Given:
\begin{itemize}
 \item \verb=omega=: wavenumber in $\rm cm^{-1}$,
 \item \verb=temp=: temperature in K,
 \item \verb=pc7h8=: toluene partial pressure in atm,
 \item \verb=ptot=: total pressure in atm,
 \item \verb=gc3=: collisional broadening half-width at half-maximum, units in $\rm cm^{-1}$,
\end{itemize}
this subroutine returns the spectral properties of toluene using new data:
\begin{itemize}
 \item \verb=sdweak=: spectral mean absorption coefficient, units in $\rm atm^{-1}.cm^{-1}$,
 \item \verb=gdinv=: the collision broadening fine structure parameter, dimensionless,
 \item \verb=gddinv=: the Doppler broadening fine structure parameter, dimensionless,
 \item \verb=i_model=: the narrow band model used.
\end{itemize}

\begin{lstlisting}
real(eb), intent(in)  :: omega, temp, pc7h8, ptot
real(eb), intent(out) :: sdweak, gdinv, gddinv

integer, intent(out) :: i_model

real(eb) :: q2ot, azot, toaz
real(eb) :: gd, pressure_effective
real(eb) :: dinv_c7h8

real(eb), parameter :: q2       = 1.4388_eb  ! q2 = speed of light*planck cns/boltzmann
real(eb), parameter :: wm_c7h8  = 92.1384_eb ! molecular weight (g/mol)

integer :: i_band, i

logical :: in_band ! true if omega is within some tabulated band, false otherwise
\end{lstlisting}


\subsection{Subroutine ch3oh}
\label{sub:ch3oh}


\begin{lstlisting}
subroutine ch3oh(omega,temp,pch3oh,ptot,sdweak,gdinv,gddinv,i_model)
\end{lstlisting}

 Given:
\begin{itemize}
 \item \verb=omega=: wavenumber in $\rm cm^{-1}$,
 \item \verb=temp=: temperature in K,
 \item \verb=pch3oh=: methanol partial pressure in atm,
 \item \verb=ptot=: total pressure in atm,
 \item \verb=gc3=: collisional broadening half-width at half-maximum, units in $\rm cm^{-1}$,
\end{itemize}
this subroutine returns the spectral properties of methanol using new data:
\begin{itemize}
 \item \verb=sdweak=: spectral mean absorption coefficient, units in $\rm atm^{-1}.cm^{-1}$,
 \item \verb=gdinv=: the collision broadening fine structure parameter, dimensionless,
 \item \verb=gddinv=: the Doppler broadening fine structure parameter, dimensionless,
 \item \verb=i_model=: the narrow band model used.
\end{itemize}

\begin{lstlisting}
real(eb), intent(in)  :: omega, temp, pch3oh, ptot
real(eb), intent(out) :: sdweak, gdinv, gddinv

integer, intent(out) :: i_model

real(eb) :: q2ot, azot, toaz
real(eb) :: gd, pressure_effective
real(eb) :: dinv_ch3oh

real(eb), parameter :: q2       = 1.4388_eb  ! q2 = speed of light*planck cns/boltzmann
real(eb), parameter :: wm_ch3oh = 32.0419_eb ! molecular weight (g/mol)
integer :: i_band, i

logical :: in_band ! true if omega is within some tabulated band, false otherwise
\end{lstlisting}

\subsection{Subroutine c5h8o2}
\label{sub:c5h8o2}

\begin{lstlisting}
subroutine c5h8o2(omega,temp,pc5h8o2,ptot,sdweak,gdinv,gddinv,i_model)
\end{lstlisting}

 Given:
\begin{itemize}
 \item \verb=omega=: wavenumber in $\rm cm^{-1}$,
 \item \verb=temp=: temperature in K,
 \item \verb=pc5h8o2=: MMA partial pressure in atm,
 \item \verb=ptot=: total pressure in atm,
 \item \verb=gc3=: collisional broadening half-width at half-maximum, units in $\rm cm^{-1}$,
\end{itemize}
this subroutine returns the spectral properties of MMA using new data:
\begin{itemize}
 \item \verb=sdweak=: spectral mean absorption coefficient, units in $\rm atm^{-1}.cm^{-1}$,
 \item \verb=gdinv=: the collision broadening fine structure parameter, dimensionless,
 \item \verb=gddinv=: the Doppler broadening fine structure parameter, dimensionless,
 \item \verb=i_model=: the narrow band model used.
\end{itemize}

\begin{lstlisting}
real(eb), intent(in)  :: omega, temp, pc5h8o2, ptot
real(eb), intent(out) :: sdweak, gdinv, gddinv

integer, intent(out) :: i_model

real(eb) :: q2ot, azot, toaz
real(eb) :: gd, pressure_effective
real(eb) :: dinv_c5h8o2

real(eb), parameter :: q2        = 1.4388_eb   ! q2 = speed of light*planck cns/boltzmann
real(eb), parameter :: wm_c5h8o2 = 100.1158_eb ! molecular weight (g/mol)
integer :: i_band, i

logical :: in_band ! true if omega is within some tabulated band, false otherwise
\end{lstlisting}

\subsection{Subroutine c2h6}
\label{sub:c2h6}
\begin{lstlisting}
subroutine c2h6(omega,temp,pc2h6,ptot,sdweak,gdinv,gddinv,i_model)
\end{lstlisting}

 Given:
\begin{itemize}
 \item \verb=omega=: wavenumber in $\rm cm^{-1}$,
 \item \verb=temp=: temperature in K,
 \item \verb=pc2h6=: ethane partial pressure in atm,
 \item \verb=ptot=: total pressure in atm,
 \item \verb=gc3=: collisional broadening half-width at half-maximum, units in $\rm cm^{-1}$,
\end{itemize}
this subroutine returns the spectral properties of ethane using new data:
\begin{itemize}
 \item \verb=sdweak=: spectral mean absorption coefficient, units in $\rm atm^{-1}.cm^{-1}$,
 \item \verb=gdinv=: the collision broadening fine structure parameter, dimensionless,
 \item \verb=gddinv=: the Doppler broadening fine structure parameter, dimensionless,
 \item \verb=i_model=: the narrow band model used.
\end{itemize}

\begin{lstlisting}
real(eb), intent(in)  :: omega, temp, pc2h6, ptot
real(eb), intent(out) :: sdweak, gdinv, gddinv

integer, intent(out) :: i_model

real(eb) :: q2ot, azot, toaz
real(eb) :: gd, pressure_effective
real(eb) :: dinv_c2h6

real(eb), parameter :: q2      = 1.4388_eb  ! q2 = speed of light*planck cns/boltzmann
real(eb), parameter :: wm_c2h6 = 30.0690_eb ! nist webbook data
integer :: i_band, i

logical :: in_band ! true if omega is within some tabulated band, false otherwise
\end{lstlisting}

\subsection{Subroutine c2h4}
\label{sub:c2h4}

\begin{lstlisting}
subroutine c2h4(omega,temp,pc2h4,ptot,sdweak,gdinv,gddinv,i_model)
\end{lstlisting}

Given:
\begin{itemize}
 \item \verb=omega=: wavenumber in $\rm cm^{-1}$,
 \item \verb=temp=: temperature in K,
 \item \verb=pc2h4=: ethylene partial pressure in atm,
 \item \verb=ptot=: total pressure in atm,
 \item \verb=gc3=: collisional broadening half-width at half-maximum, units in $\rm cm^{-1}$,
\end{itemize}
this subroutine returns the spectral properties of ethylene using new data:
\begin{itemize}
 \item \verb=sdweak=: spectral mean absorption coefficient, units in $\rm atm^{-1}.cm^{-1}$,
 \item \verb=gdinv=: the collision broadening fine structure parameter, dimensionless,
 \item \verb=gddinv=: the Doppler broadening fine structure parameter, dimensionless,
 \item \verb=i_model=: the narrow band model used.
\end{itemize}

\begin{lstlisting}
real(eb), intent(in)  :: omega, temp, pc2h4, ptot
real(eb), intent(out) :: sdweak, gdinv, gddinv

integer, intent(out) :: i_model

real(eb) :: q2ot, azot, toaz
real(eb) :: gd, pressure_effective
real(eb) :: dinv_c2h4

real(eb), parameter :: q2      = 1.4388_eb  ! q2 = speed of light*planck cns/boltzmann
real(eb), parameter :: wm_c2h4 = 28.0532_eb ! nist webbook data
integer :: i_band, i

logical :: in_band ! true if omega is within some tabulated band, false otherwise
\end{lstlisting}

\subsection{Function get\_spectral\_absorption}

\begin{lstlisting}
pure real(eb) function get_spectral_absorption(omega,temp,sd_temp,bounds,absorption_data)
\end{lstlisting}
This function returns interpolated (in temperature and wavenumber) values of spectral absorption.

Variables in:
\begin{itemize}
 \item \verb=omega=  : wavenumber, units in $\rm cm^{-1}$
 \item \verb=temp=   : gas temperature, units in K
 \item \verb=sd_temp=: vector of temperature measurements, units in K
 \item \verb=bounds= : vector of wavenumber bounds and spacing for a given band, \verb=dimensions(3)=
 \item \verb=absorption_data=: array of species spectral measurements. Note: first dimension equals to dimension of \verb=sd_temp=.
\end{itemize}

Action: Bilinear interpolation of \verb=absorption_data=.

\begin{lstlisting}
 real(eb), intent(in) :: omega
 real(eb), intent(in) :: temp
 real(eb), dimension(:), intent(in)   :: sd_temp, bounds
 real(eb), dimension(:,:), intent(in) :: absorption_data

 real(eb) :: delta_t
 real(eb) :: delta_w
 real(eb) :: w1
 real(eb) :: omega_min, omega_max, delta_omega
 real(eb) :: ttemp
 real(eb) :: interpolated_value

 integer :: n_temp, n_bounds, n_abs_data1, n_abs_data2
 integer :: i_omega, i_temp, i_omega1, i_omega2

 logical :: cross
\end{lstlisting}

\subsection{Function planck}

\begin{lstlisting}
elemental real(eb) function planck(temp,lambda)
\end{lstlisting}
Computes blackbody distribution at wavelength \verb=lambda= and at temperature \verb=temp=. Values returned in units of $\rm W/m^{2}/str/cm^{-1}$. Solves Eq.~\ref{eq:Planck_WL}.

Variables in:
\begin{itemize}
 \item \verb=temp=: temperature, units in K
 \item \verb=lambda=: wavelength, units in $\rm \mu m$
\end{itemize}

\begin{lstlisting}
 real (eb), intent(in):: temp, lambda
 real (eb), parameter :: q1 = 1.19088e8_eb ! q1 = 2*speed of light$^2$*planck\_cnst
 real (eb), parameter :: q2 = 14388._eb    ! q2 = speed of light*planck cns/boltzmann
 real (eb) :: c   ! c = lambda * temp
\end{lstlisting}

\subsection{Function planck\_wn}

\begin{lstlisting}
elemental real(eb) function planck_wn(temp,omega)
\end{lstlisting}
Computes blackbody distribution at wavenumber \verb=omega= and at temperature \verb=temp=. Values returned in units of $\rm W/m^{2}/str/cm^{-1}$. Solves Eq.~\ref{eq:Planck_WL}.

Variables in:
\begin{itemize}
 \item \verb=temp=: temperature, units in K
 \item \verb=omega=: wavenumber, units in $\rm cm^{-1}$
\end{itemize}

\begin{lstlisting}
 real (eb), intent(in):: temp, omega

 real (eb), parameter :: q1 = 1.19088e-8_eb ! q1 = 2*speed of light$^2$*planck\_cnst
 real (eb), parameter :: q2 = 1.4388_eb
 real (eb) :: c   ! c = temp/omega
\end{lstlisting}


\subsection{Function integration}
\label{fun:integration}
\begin{lstlisting}
real(eb) function integration(x,y,io)
\end{lstlisting}
 Performs numerical integration of the vector \verb=y= over the range given by vector \verb=x=. It does not assume any regularity of \verb=x=. The integration is based on Simpson rule over non regular abscissa, uses Lagrangian interpolation using 3 points (quadratic interpolation). Note: \verb=io= is the unit of output file, which is needed for printing error message.

\begin{lstlisting}
 real(eb), intent(in), dimension(:) :: x,y
 integer,  intent(in) :: io

 real(eb) :: alpha, beta, gamma, b, a, segment_int
 integer  :: i_point, n_points, i, j ,k
\end{lstlisting}


\subsection{Subroutine rcalloc}

\begin{lstlisting}

\end{lstlisting}

This subroutine allocates memory and assigns the variables containing the spectral properties of the species present in RadCal.

\subsection{Subroutine populate\_species}
\label{sub:populate_species}

\begin{lstlisting}
subroutine populate_species(io)
\end{lstlisting}
 This function populates the variable \verb=species= that contains the RadCal name and the name of the species present in gas phase.

\begin{lstlisting}
 integer, intent(in) :: io
\end{lstlisting}


\subsection{Function index\_species}
\label{fun:index_species}

\begin{lstlisting}
function index_species(molecule) result(i_molecule)
\end{lstlisting}

This function returns the index of the molecule presents in the variable \verb=species=.
\verb=i_molecule= is such that:\\
 molecule = species(i\_molecule)\%id or\\
 molecule = species(i\_molecule)\%radcal\_id.

\begin{lstlisting}
 character(len=*) :: molecule

 integer :: i_molecule
 integer :: i_species
 logical :: found

\end{lstlisting}


\subsection{Subroutine rcdealloc}

\begin{lstlisting}
subroutine rcdealloc
\end{lstlisting}
Deallocates arrays variables. Performed prior to RadCal exit.

\subsection{Subroutine termination}
\label{sub:termination}
\begin{lstlisting}
subroutine termination(ierr,io)
\end{lstlisting}
 Subroutine called when exceptions are raised. Terminates the program and writes error messages depending on the context.

 Variables passed in:
 \begin{itemize}
  \item  \verb=ierr= : error message indice
  \item  \verb=io=   : file unit number.
 \end{itemize}

\begin{lstlisting}
 integer, intent(in) :: ierr, io
 character(len=2056) :: message
\end{lstlisting}


\subsection{Subroutine write\_input}
\label{sub:write_input}

\begin{lstlisting}
subroutine write_input(io)
\end{lstlisting}
 This subroutine writes a default \verb=RADCAL.in= in the case that no \verb=RADCAL.in= is provided.
 In particular, it writes the available species. This subroutine should only be called when \verb=RADCAL.in= does not exist.
\begin{lstlisting}
 integer, intent(in) :: io

 integer, parameter  :: i_input = 10
 integer             :: ierr
 character(len=30)   :: filename

 character(len=2048) :: header
 character(len=2048) :: line_text
 character           :: character_line(80)

 logical             :: file_exist

 integer :: i_species, i

\end{lstlisting}

\subsection{Subroutine read\_input}
\label{sub:read_input}

\begin{lstlisting}
subroutine read_input(io)
\end{lstlisting}
 This subroutine reads the input file \verb=RADCAL.in= (unit \verb=io=).

\begin{lstlisting}
 integer, intent(in) :: io

 integer           :: i_input, ierr
 character(len=30) :: filename

 logical           :: file_exist

\end{lstlisting}

\subsection{Subroutine read\_point}
\label{sub:read_point}

\begin{lstlisting}
subroutine read_point(i_input,i_output)
\end{lstlisting}
 This subroutine reads the input file \verb=RADCAL.in= (unit \verb=i_input=) and searches for the keyword \verb=path_segment=. It counts the number of homogeneous segments that comprise a pathline, and assigns user defined values of species mole fractions or volume fraction (only for soot), temperature (assumed to be uniform along a given homogeneous segment of the pathline), length \verb=length=, and pressure (in atm) to all the segments.

 Units:
 \begin{itemize}
  \item \verb=T=, temperature, in K
  \item \verb=LENGTH=, homogeneous segment length, in $m$
  \item species data (\verb=x<species>=), in mole fraction
  \item \verb=pressure=, in atm
  \item \verb=fv=, soot volume fraction, dimensionless.
 \end{itemize}

\begin{lstlisting}
 integer, intent(in) :: i_input  ! unit of the input file. alreay opened
 integer, intent(in) :: i_output ! unit of the output file. alreay opened

 type gas_phase
    real(kind=eb), pointer :: obj
    character(len=30)      :: name
 end type gas_phase

 type(gas_phase), allocatable, dimension(:) ::  Mole_Fraction

 real(kind=eb) :: sum_mole_fraction

 real(kind=eb) :: t, length, pressure, fv

 real(kind=eb), target :: xco2, xh2o, xco, xch4, xc2h4, xc2h6, xc3h8,xc3h6, xc7h8, xc7h16, xch3oh, xmma, xn2, xch4_old, xo2

 character(len=255) :: line

 integer :: n_segment, i_segment, i_species, i_species_gas, status, i_fv

\end{lstlisting}


\subsection{Subroutine read\_band}
\label{sub:read_band}
\begin{lstlisting}
subroutine read_band(i_input,i_output)
\end{lstlisting}
This subroutine reads the input file \verb=RADCAL.in= (unit \verb=i_input=) and searches for the keyword \verb=&BANDS=. \verb=&BANDS= defines the lower and upper bound of the spectrum to be computed, \verb=ommin= and \verb=ommax=, respectively. Both are given in cm$^{-1}$.

\begin{lstlisting}
 integer, intent(in) :: i_input  ! unit of the input file. alreay opened
 integer, intent(in) :: i_output ! unit of the output file. alreay opened

 integer :: io_err ! error condition number
\end{lstlisting}


\subsection{Subroutine read\_wall}
\label{sub:read_wall}
\begin{lstlisting}
subroutine read_wall(i_input,i_output)
\end{lstlisting}
 This subroutine reads the input file \verb=RADCAL.in= (unit \verb=i_input=) and searches for the keyword \verb=&WALL= which defines the wall (or infinity) temperature: \verb=TWALL=. In RadCal, the wall emits like a blackbody of temperature \verb=TWALL=.

\begin{lstlisting}
 integer, intent(in) :: i_input  ! unit of the input file. alreay opened
 integer, intent(in) :: i_output ! unit of the output file. alreay opened

 integer :: io_err ! catch the value of the error raised by output subroutine
\end{lstlisting}


\subsection{Subroutine read\_header}
\label{sub:read_header}

\begin{lstlisting}
subroutine read_header(i_input,i_output)
\end{lstlisting}
This subroutine reads the input file \verb=RADCAL.in= (unit \verb=i_input=) and searches for the keyword \verb=&HEADER=. \verb=&HEADER= defines the case id (\verb=CHID=) used to generate the output Tecplot file and the case title (\verb=TITLE=), which is to be printed in the output file.

\begin{lstlisting}
 integer, intent(in) :: i_input  ! unit of the input file. alreay opened
 integer, intent(in) :: i_output ! unit of the output file. alreay opened

 integer :: io_err ! catch the value of the error raised by output subroutine
\end{lstlisting}

\section{Driver (main) Program}
\subsection{Program driver}
\label{prog:driver}


\begin{lstlisting}
 use radcal

\end{lstlisting}

This is the main program. It operates the different RadCal functions that read the input file, allocate the needed variables, solve the RTE, and print the results. Note: the type \verb=real(exb)= is used for intrinsic timer routine \verb=cpu_time=. The floating type \verb=eb= is defined in the \verb=radcal= module.

\begin{lstlisting}
 integer, parameter :: exb = selected_real_kind(16)

 real(eb), allocatable, dimension(:) :: transmissivity

 real(eb) :: amean, planck_mean_absorption, flux, total_length_m,total_length_cm, total_transmissivity

 integer :: io
 character(len=255) :: filename

 real(exb) :: time_init
 real(exb) :: time_end
\end{lstlisting}

\subsection{Subroutine tau\_print}
\label{sub:tau_print}

\begin{lstlisting}
subroutine tau_print(case_id, pressure, path_length, transmissivity, wave_length)
\end{lstlisting}
This function prints, in ASCII format, the wavenumber in $\rm cm^{-1}$, the transmissivity in \%, and the incident radiance in $\rm W/m^{2}/str/cm^{-1}$, in the file \verb=<case_id>.tec=.

\begin{lstlisting}
 use radcal, only : eb, incident_radiance
\end{lstlisting}


\begin{lstlisting}
 real(eb), dimension(:), intent(in) :: wave_length    ! Wave length in micron
 real(eb), dimension(:), intent(in) :: transmissivity ! Transmissivity
 real(eb), intent(in) :: pressure
 real(eb), intent(in) :: path_length

 character(len=255), intent(in) :: case_id

 character(len=15)  :: pressure_atm
 character(len=15)  :: path_length_m
 character(len=255) :: filename
 character(len=50)  :: format_output

 integer :: n_max
 integer :: i_tecfile, i
\end{lstlisting}
